\documentclass[openany]{book}
\usepackage{lmodern}
\usepackage{amssymb,amsmath}
\usepackage{ifxetex,ifluatex}
\usepackage{fixltx2e} % provides \textsubscript
\ifnum 0\ifxetex 1\fi\ifluatex 1\fi=0 % if pdftex
  \usepackage[T1]{fontenc}
  \usepackage[utf8]{inputenc}
\else % if luatex or xelatex
  \ifxetex
    \usepackage{mathspec}
  \else
    \usepackage{fontspec}
  \fi
  \defaultfontfeatures{Ligatures=TeX,Scale=MatchLowercase}
\fi
% use upquote if available, for straight quotes in verbatim environments
\IfFileExists{upquote.sty}{\usepackage{upquote}}{}
% use microtype if available
\IfFileExists{microtype.sty}{%
\usepackage{microtype}
\UseMicrotypeSet[protrusion]{basicmath} % disable protrusion for tt fonts
}{}
\usepackage{hyperref}
\hypersetup{unicode=true,
            pdftitle={DATA 624: Project 1},
            pdfauthor={Vinicio Haro; Sang Yoon (Andy) Hwang; Julian McEachern; Jeremy O'Brien; Bethany Poulin},
            pdfborder={0 0 0},
            breaklinks=true}
\urlstyle{same}  % don't use monospace font for urls
\usepackage{natbib}
\bibliographystyle{plainnat}
\usepackage{color}
\usepackage{fancyvrb}
\newcommand{\VerbBar}{|}
\newcommand{\VERB}{\Verb[commandchars=\\\{\}]}
\DefineVerbatimEnvironment{Highlighting}{Verbatim}{commandchars=\\\{\}}
% Add ',fontsize=\small' for more characters per line
\usepackage{framed}
\definecolor{shadecolor}{RGB}{248,248,248}
\newenvironment{Shaded}{\begin{snugshade}}{\end{snugshade}}
\newcommand{\AlertTok}[1]{\textcolor[rgb]{0.94,0.16,0.16}{#1}}
\newcommand{\AnnotationTok}[1]{\textcolor[rgb]{0.56,0.35,0.01}{\textbf{\textit{#1}}}}
\newcommand{\AttributeTok}[1]{\textcolor[rgb]{0.77,0.63,0.00}{#1}}
\newcommand{\BaseNTok}[1]{\textcolor[rgb]{0.00,0.00,0.81}{#1}}
\newcommand{\BuiltInTok}[1]{#1}
\newcommand{\CharTok}[1]{\textcolor[rgb]{0.31,0.60,0.02}{#1}}
\newcommand{\CommentTok}[1]{\textcolor[rgb]{0.56,0.35,0.01}{\textit{#1}}}
\newcommand{\CommentVarTok}[1]{\textcolor[rgb]{0.56,0.35,0.01}{\textbf{\textit{#1}}}}
\newcommand{\ConstantTok}[1]{\textcolor[rgb]{0.00,0.00,0.00}{#1}}
\newcommand{\ControlFlowTok}[1]{\textcolor[rgb]{0.13,0.29,0.53}{\textbf{#1}}}
\newcommand{\DataTypeTok}[1]{\textcolor[rgb]{0.13,0.29,0.53}{#1}}
\newcommand{\DecValTok}[1]{\textcolor[rgb]{0.00,0.00,0.81}{#1}}
\newcommand{\DocumentationTok}[1]{\textcolor[rgb]{0.56,0.35,0.01}{\textbf{\textit{#1}}}}
\newcommand{\ErrorTok}[1]{\textcolor[rgb]{0.64,0.00,0.00}{\textbf{#1}}}
\newcommand{\ExtensionTok}[1]{#1}
\newcommand{\FloatTok}[1]{\textcolor[rgb]{0.00,0.00,0.81}{#1}}
\newcommand{\FunctionTok}[1]{\textcolor[rgb]{0.00,0.00,0.00}{#1}}
\newcommand{\ImportTok}[1]{#1}
\newcommand{\InformationTok}[1]{\textcolor[rgb]{0.56,0.35,0.01}{\textbf{\textit{#1}}}}
\newcommand{\KeywordTok}[1]{\textcolor[rgb]{0.13,0.29,0.53}{\textbf{#1}}}
\newcommand{\NormalTok}[1]{#1}
\newcommand{\OperatorTok}[1]{\textcolor[rgb]{0.81,0.36,0.00}{\textbf{#1}}}
\newcommand{\OtherTok}[1]{\textcolor[rgb]{0.56,0.35,0.01}{#1}}
\newcommand{\PreprocessorTok}[1]{\textcolor[rgb]{0.56,0.35,0.01}{\textit{#1}}}
\newcommand{\RegionMarkerTok}[1]{#1}
\newcommand{\SpecialCharTok}[1]{\textcolor[rgb]{0.00,0.00,0.00}{#1}}
\newcommand{\SpecialStringTok}[1]{\textcolor[rgb]{0.31,0.60,0.02}{#1}}
\newcommand{\StringTok}[1]{\textcolor[rgb]{0.31,0.60,0.02}{#1}}
\newcommand{\VariableTok}[1]{\textcolor[rgb]{0.00,0.00,0.00}{#1}}
\newcommand{\VerbatimStringTok}[1]{\textcolor[rgb]{0.31,0.60,0.02}{#1}}
\newcommand{\WarningTok}[1]{\textcolor[rgb]{0.56,0.35,0.01}{\textbf{\textit{#1}}}}
\usepackage{graphicx,grffile}
\makeatletter
\def\maxwidth{\ifdim\Gin@nat@width>\linewidth\linewidth\else\Gin@nat@width\fi}
\def\maxheight{\ifdim\Gin@nat@height>\textheight\textheight\else\Gin@nat@height\fi}
\makeatother
% Scale images if necessary, so that they will not overflow the page
% margins by default, and it is still possible to overwrite the defaults
% using explicit options in \includegraphics[width, height, ...]{}
\setkeys{Gin}{width=\maxwidth,height=\maxheight,keepaspectratio}
\IfFileExists{parskip.sty}{%
\usepackage{parskip}
}{% else
\setlength{\parindent}{0pt}
\setlength{\parskip}{6pt plus 2pt minus 1pt}
}
\setlength{\emergencystretch}{3em}  % prevent overfull lines
\providecommand{\tightlist}{%
  \setlength{\itemsep}{0pt}\setlength{\parskip}{0pt}}
\setcounter{secnumdepth}{5}

%%% Use protect on footnotes to avoid problems with footnotes in titles
\let\rmarkdownfootnote\footnote%
\def\footnote{\protect\rmarkdownfootnote}

%%% Change title format to be more compact
\usepackage{titling}

% Create subtitle command for use in maketitle
\providecommand{\subtitle}[1]{
  \posttitle{
    \begin{center}\large#1\end{center}
    }
}

\setlength{\droptitle}{-2em}

  \title{DATA 624: Project 1}
    \pretitle{\vspace{\droptitle}\centering\huge}
  \posttitle{\par}
    \author{Vinicio Haro \\ Sang Yoon (Andy) Hwang \\ Julian McEachern \\ Jeremy O'Brien \\ Bethany Poulin}
    \preauthor{\centering\large\emph}
  \postauthor{\par}
      \predate{\centering\large\emph}
  \postdate{\par}
    \date{October 22, 2019}

\usepackage{booktabs}
\usepackage[table]{xcolor}

% set plain style for page numbers
\pagestyle{plain}
\raggedbottom

% change font
\usepackage{fontspec}
\setmainfont{Arial}

% remove "chapter" from chapter title
\usepackage{titlesec}
\titleformat{\chapter}
  {\normalfont\LARGE\bfseries}{\thechapter}{1em}{}
\titlespacing*{\chapter}{0pt}{3.5ex plus 1ex minus .2ex}{2.3ex plus .2ex}

% create color block quotes
\usepackage{tcolorbox}
\newtcolorbox{myquote}{colback=orange!05!white, colframe=black!75!black}
\renewenvironment{quote}{\begin{myquote}}{\end{myquote}}

% wrap text
\usepackage{geometry}[textwidth=6in]

% kable 
\usepackage{tabu}
\usepackage{float}
\usepackage{booktabs}
\usepackage{longtable}
\usepackage{array}
\usepackage{multirow}
\usepackage{wrapfig}
\usepackage{float}
\usepackage{colortbl}
\usepackage{pdflscape}
\usepackage{tabu}
\usepackage{threeparttable}
\usepackage{threeparttablex}
\usepackage[normalem]{ulem}
\usepackage{makecell}
\usepackage{xcolor}

\begin{document}
\maketitle

{
\setcounter{tocdepth}{1}
\tableofcontents
}
\hypertarget{overview}{%
\chapter*{Overview}\label{overview}}
\addcontentsline{toc}{chapter}{Overview}

\begin{quote}
Insert Project Overview and explain work process.
\end{quote}

\hypertarget{dependencies}{%
\section*{Dependencies}\label{dependencies}}
\addcontentsline{toc}{section}{Dependencies}

\begin{quote}
Please add all libraries used here.
\end{quote}

The following R libraries were used to complete Project 1:

\begin{Shaded}
\begin{Highlighting}[]
\CommentTok{# Insert All Used Dependencies Here }
\end{Highlighting}
\end{Shaded}

\hypertarget{data}{%
\section*{Data}\label{data}}
\addcontentsline{toc}{section}{Data}

Data was stored within our group repository and imported below using the
\texttt{readxl} package. Each individual question was solved within an R
script and the data was sourced into our main report for discussion
purposes. The R scripts are available within our appendix for
replication purposes.

For grading purposes, we exported and saved all forecasts as a csv in
our data folder.

\begin{Shaded}
\begin{Highlighting}[]
\CommentTok{# Data Aquisition}
\NormalTok{atm_data <-}\StringTok{ }\KeywordTok{read_excel}\NormalTok{(}\StringTok{"data/ATM624Data.xlsx"}\NormalTok{) }
\NormalTok{power_data <-}\StringTok{ }\KeywordTok{read_excel}\NormalTok{(}\StringTok{"data/ResidentialCustomerForecastLoad-624.xlsx"}\NormalTok{) }
\NormalTok{pipe1_data <-}\StringTok{ }\KeywordTok{read_excel}\NormalTok{(}\StringTok{"data/Waterflow_Pipe1.xlsx"}\NormalTok{)}
\NormalTok{pipe2_data <-}\StringTok{ }\KeywordTok{read_excel}\NormalTok{(}\StringTok{"data/Waterflow_Pipe2.xlsx"}\NormalTok{)}

\CommentTok{# Source Code}
\KeywordTok{source}\NormalTok{(}\StringTok{"scripts/Part-A-JM.R"}\NormalTok{) }\CommentTok{## Example}
\end{Highlighting}
\end{Shaded}

\hypertarget{part-a-atms}{%
\chapter{Part A: ATMs}\label{part-a-atms}}

\begin{quote}
\textbf{Instructions:} In part A, I want you to forecast how much cash
is taken out of 4 different ATM machines for May 2010. The data is given
in a single file. The variable \texttt{Cash} is provided in hundreds of
dollars, other than that it is straight forward. I am being somewhat
ambiguous on purpose. I am giving you data, please provide your written
report on your findings, visuals, discussion and your R code all within
a Word readable document, except the forecast which you will put in an
Excel readable file. I must be able to cut and paste your R code and run
it in R studio. Your report must be professional - most of all -
readable, EASY to follow. Let me know what you are thinking, assumptions
you are making! Your forecast is a simple CSV or Excel file that MATCHES
the format of the data I provide.
\end{quote}

\hypertarget{part-b-power}{%
\chapter{Part B: Power}\label{part-b-power}}

\begin{quote}
\textbf{Instructions:} Forecasting Power: Part B consists of a simple
dataset of residential power usage for January 1998 until December 2013.
Your assignment is to model these data and a monthly forecast for 2014.
The data is given in a single file. The variable `KWH' is power
consumption in Kilowatt hours, the rest is straight forward. Add these
to your existing files above - clearly labeled.
\end{quote}

\hypertarget{part-c-waterflow}{%
\chapter{Part C: Waterflow}\label{part-c-waterflow}}

\begin{quote}
\textbf{Instructions:} Part C consists of two data sets. These are
simple 2 columns sets, however they have different time stamps. Your
optional assignment is to time-base sequence the data and aggregate
based on hour (example of what this looks like, follows). Note for
multiple recordings within an hour, take the mean. Then to test
appropriate assumptions and forecast a week forward with confidence
bands (80 and 95\%). Add these to your existing files above - clearly
labeled.
\end{quote}

\hypertarget{Appendix}{%
\chapter*{Appendix}\label{Appendix}}
\addcontentsline{toc}{chapter}{Appendix}

\hypertarget{Part-A}{%
\section*{Part A}\label{Part-A}}
\addcontentsline{toc}{section}{Part A}

\hypertarget{Part-A-FC1}{%
\subsection*{ATM1 Forecast}\label{Part-A-FC1}}
\addcontentsline{toc}{subsection}{ATM1 Forecast}

\hypertarget{Part-A-FC2}{%
\subsection*{ATM2 Forecast}\label{Part-A-FC2}}
\addcontentsline{toc}{subsection}{ATM2 Forecast}

\hypertarget{Part-A-FC3}{%
\subsection*{ATM3 Forecast}\label{Part-A-FC3}}
\addcontentsline{toc}{subsection}{ATM3 Forecast}

\hypertarget{Part-A-FC4}{%
\subsection*{ATM4 Forecast}\label{Part-A-FC4}}
\addcontentsline{toc}{subsection}{ATM4 Forecast}

\hypertarget{Part-A-RScript}{%
\subsection*{R Script}\label{Part-A-RScript}}
\addcontentsline{toc}{subsection}{R Script}

\begin{Shaded}
\begin{Highlighting}[]
\CommentTok{#Insert Script Here }
\end{Highlighting}
\end{Shaded}

\hypertarget{Part-B}{%
\section*{Part B}\label{Part-B}}
\addcontentsline{toc}{section}{Part B}

\hypertarget{Part-B-FC1}{%
\subsection*{Power Forecast}\label{Part-B-FC1}}
\addcontentsline{toc}{subsection}{Power Forecast}

\hypertarget{Part-B-RScript}{%
\subsection*{R Script}\label{Part-B-RScript}}
\addcontentsline{toc}{subsection}{R Script}

\begin{Shaded}
\begin{Highlighting}[]
\CommentTok{#Insert Script Here }
\end{Highlighting}
\end{Shaded}

\hypertarget{Part-C}{%
\section*{Part C}\label{Part-C}}
\addcontentsline{toc}{section}{Part C}

\hypertarget{Part-C-FC1}{%
\subsection*{Pipes1 Forecast}\label{Part-C-FC1}}
\addcontentsline{toc}{subsection}{Pipes1 Forecast}

\hypertarget{Part-C-FC2}{%
\subsection*{Pipes2 Forecast}\label{Part-C-FC2}}
\addcontentsline{toc}{subsection}{Pipes2 Forecast}

\hypertarget{Part-C-RScript}{%
\subsection*{R Script}\label{Part-C-RScript}}
\addcontentsline{toc}{subsection}{R Script}

\begin{Shaded}
\begin{Highlighting}[]
\CommentTok{#Insert Script Here }
\end{Highlighting}
\end{Shaded}


\end{document}
