\documentclass[]{report}
\usepackage{lmodern}
\usepackage{amssymb,amsmath}
\usepackage{ifxetex,ifluatex}
\usepackage{fixltx2e} % provides \textsubscript
\ifnum 0\ifxetex 1\fi\ifluatex 1\fi=0 % if pdftex
  \usepackage[T1]{fontenc}
  \usepackage[utf8]{inputenc}
\else % if luatex or xelatex
  \ifxetex
    \usepackage{mathspec}
  \else
    \usepackage{fontspec}
  \fi
  \defaultfontfeatures{Ligatures=TeX,Scale=MatchLowercase}
\fi
% use upquote if available, for straight quotes in verbatim environments
\IfFileExists{upquote.sty}{\usepackage{upquote}}{}
% use microtype if available
\IfFileExists{microtype.sty}{%
\usepackage{microtype}
\UseMicrotypeSet[protrusion]{basicmath} % disable protrusion for tt fonts
}{}
\usepackage[margin=1in]{geometry}
\usepackage{hyperref}
\hypersetup{unicode=true,
            pdftitle={DATA 624: Project 2},
            pdfauthor={Vinicio Haro; Sang Yoon (Andy) Hwang; Julian McEachern; Jeremy O'Brien; Bethany Poulin},
            pdfborder={0 0 0},
            breaklinks=true}
\urlstyle{same}  % don't use monospace font for urls
\usepackage{color}
\usepackage{fancyvrb}
\newcommand{\VerbBar}{|}
\newcommand{\VERB}{\Verb[commandchars=\\\{\}]}
\DefineVerbatimEnvironment{Highlighting}{Verbatim}{commandchars=\\\{\}}
% Add ',fontsize=\small' for more characters per line
\usepackage{framed}
\definecolor{shadecolor}{RGB}{248,248,248}
\newenvironment{Shaded}{\begin{snugshade}}{\end{snugshade}}
\newcommand{\KeywordTok}[1]{\textcolor[rgb]{0.13,0.29,0.53}{\textbf{#1}}}
\newcommand{\DataTypeTok}[1]{\textcolor[rgb]{0.13,0.29,0.53}{#1}}
\newcommand{\DecValTok}[1]{\textcolor[rgb]{0.00,0.00,0.81}{#1}}
\newcommand{\BaseNTok}[1]{\textcolor[rgb]{0.00,0.00,0.81}{#1}}
\newcommand{\FloatTok}[1]{\textcolor[rgb]{0.00,0.00,0.81}{#1}}
\newcommand{\ConstantTok}[1]{\textcolor[rgb]{0.00,0.00,0.00}{#1}}
\newcommand{\CharTok}[1]{\textcolor[rgb]{0.31,0.60,0.02}{#1}}
\newcommand{\SpecialCharTok}[1]{\textcolor[rgb]{0.00,0.00,0.00}{#1}}
\newcommand{\StringTok}[1]{\textcolor[rgb]{0.31,0.60,0.02}{#1}}
\newcommand{\VerbatimStringTok}[1]{\textcolor[rgb]{0.31,0.60,0.02}{#1}}
\newcommand{\SpecialStringTok}[1]{\textcolor[rgb]{0.31,0.60,0.02}{#1}}
\newcommand{\ImportTok}[1]{#1}
\newcommand{\CommentTok}[1]{\textcolor[rgb]{0.56,0.35,0.01}{\textit{#1}}}
\newcommand{\DocumentationTok}[1]{\textcolor[rgb]{0.56,0.35,0.01}{\textbf{\textit{#1}}}}
\newcommand{\AnnotationTok}[1]{\textcolor[rgb]{0.56,0.35,0.01}{\textbf{\textit{#1}}}}
\newcommand{\CommentVarTok}[1]{\textcolor[rgb]{0.56,0.35,0.01}{\textbf{\textit{#1}}}}
\newcommand{\OtherTok}[1]{\textcolor[rgb]{0.56,0.35,0.01}{#1}}
\newcommand{\FunctionTok}[1]{\textcolor[rgb]{0.00,0.00,0.00}{#1}}
\newcommand{\VariableTok}[1]{\textcolor[rgb]{0.00,0.00,0.00}{#1}}
\newcommand{\ControlFlowTok}[1]{\textcolor[rgb]{0.13,0.29,0.53}{\textbf{#1}}}
\newcommand{\OperatorTok}[1]{\textcolor[rgb]{0.81,0.36,0.00}{\textbf{#1}}}
\newcommand{\BuiltInTok}[1]{#1}
\newcommand{\ExtensionTok}[1]{#1}
\newcommand{\PreprocessorTok}[1]{\textcolor[rgb]{0.56,0.35,0.01}{\textit{#1}}}
\newcommand{\AttributeTok}[1]{\textcolor[rgb]{0.77,0.63,0.00}{#1}}
\newcommand{\RegionMarkerTok}[1]{#1}
\newcommand{\InformationTok}[1]{\textcolor[rgb]{0.56,0.35,0.01}{\textbf{\textit{#1}}}}
\newcommand{\WarningTok}[1]{\textcolor[rgb]{0.56,0.35,0.01}{\textbf{\textit{#1}}}}
\newcommand{\AlertTok}[1]{\textcolor[rgb]{0.94,0.16,0.16}{#1}}
\newcommand{\ErrorTok}[1]{\textcolor[rgb]{0.64,0.00,0.00}{\textbf{#1}}}
\newcommand{\NormalTok}[1]{#1}
\usepackage{graphicx,grffile}
\makeatletter
\def\maxwidth{\ifdim\Gin@nat@width>\linewidth\linewidth\else\Gin@nat@width\fi}
\def\maxheight{\ifdim\Gin@nat@height>\textheight\textheight\else\Gin@nat@height\fi}
\makeatother
% Scale images if necessary, so that they will not overflow the page
% margins by default, and it is still possible to overwrite the defaults
% using explicit options in \includegraphics[width, height, ...]{}
\setkeys{Gin}{width=\maxwidth,height=\maxheight,keepaspectratio}
\IfFileExists{parskip.sty}{%
\usepackage{parskip}
}{% else
\setlength{\parindent}{0pt}
\setlength{\parskip}{6pt plus 2pt minus 1pt}
}
\setlength{\emergencystretch}{3em}  % prevent overfull lines
\providecommand{\tightlist}{%
  \setlength{\itemsep}{0pt}\setlength{\parskip}{0pt}}
\setcounter{secnumdepth}{0}

%%% Use protect on footnotes to avoid problems with footnotes in titles
\let\rmarkdownfootnote\footnote%
\def\footnote{\protect\rmarkdownfootnote}

%%% Change title format to be more compact
\usepackage{titling}

% Create subtitle command for use in maketitle
\newcommand{\subtitle}[1]{
  \posttitle{
    \begin{center}\large#1\end{center}
    }
}

\setlength{\droptitle}{-2em}

  \title{DATA 624: Project 2}
    \pretitle{\vspace{\droptitle}\centering\huge}
  \posttitle{\par}
    \author{Vinicio Haro \\ Sang Yoon (Andy) Hwang \\ Julian McEachern \\ Jeremy O'Brien \\ Bethany Poulin}
    \preauthor{\centering\large\emph}
  \postauthor{\par}
      \predate{\centering\large\emph}
  \postdate{\par}
    \date{10 December 2019}

% set plain style for page numbers
\usepackage[margin=1in]{geometry}
\usepackage{fancyhdr}
\pagestyle{fancy}
\fancyhead[LE,RO]{\textbf{Group 2}}
\fancyhead[RE,LO]{\textbf{Project 2: Predicting PH}}
\raggedbottom
\setlength{\parskip}{1em}

% change font
\usepackage{fontspec}
\setmainfont{Arial}

% format titles 
\usepackage{xcolor}
\usepackage{sectsty}
\usepackage{etoolbox}
\usepackage{titling}
\definecolor{prettyblue}{RGB}{84, 144, 240}
\definecolor{bluegray}{RGB}{98, 107, 115}
\pretitle{\begin{center}\Huge\color{prettyblue}\textbf}
\posttitle{\par\LARGE\color{gray}DATA 624 - Predictive Analytics\linebreak Group 2\end{center}}
\preauthor{\begin{center}\large\textbf{Group Members:}\linebreak\textit}
\postauthor{\end{center}}

% Format chapter output
\usepackage{titlesec}
\titleclass{\part}{top}
\titleclass{\chapter}{straight}
\titleformat{\chapter}
  {\normalfont\color{prettyblue}\LARGE\bfseries}{\thechapter}{1em}{}
\titlespacing*{\chapter}{0pt}{3.5ex plus 1ex minus .2ex}{2.3ex plus .2ex}


% create color block quotes
\usepackage{tcolorbox}
\newtcolorbox{myquote}{colback=purple!05!white, colframe=purple!75!black}
\renewenvironment{quote}{\begin{myquote}}{\end{myquote}}

% kable 
\usepackage{tabu}


% multicolumn
\usepackage{multicol}

% bullets
\newenvironment{tight_enumerate}{
\begin{enumerate}
  \setlength{\itemsep}{0pt}
  \setlength{\parskip}{0pt}
  }{\end{enumerate}}
  
\newenvironment{tight_itemize}{
\begin{itemize}
  \setlength{\topsep}{0pt}
  \setlength{\itemsep}{0pt}
  \setlength{\parskip}{0pt}
  \setlength{\parsep}{0pt}
  }{\end{itemize}}

\usepackage{paralist}

%hyperlink
\usepackage{hyperref}
\hypersetup{
    colorlinks=true,
    linkcolor=bluegray,
    filecolor=magenta,      
    urlcolor=cyan}

\usepackage{graphicx}
\usepackage{wrapfig}
\usepackage{booktabs}
\definecolor{yale}{RGB}{13,77,146}
\usepackage[font={color=yale,bf,scriptsize},figurename=Fig.,belowskip=0pt,aboveskip=0pt]{caption}
\usepackage{floatrow}
\floatsetup[figure]{capposition=above}
\floatsetup[table]{capposition=above}
\setlength{\abovecaptionskip}{1pt}
\setlength{\belowcaptionskip}{1pt}
\setlength{\textfloatsep}{2pt plus 0.5pt minus 0.5pt}
\setlength{\intextsep}{2pt plus 0.5pt minus 0.5pt}

\begin{document}
\maketitle

{
\setcounter{tocdepth}{1}
\tableofcontents
}
\chapter*{Appendix}\label{Appendix}
\addcontentsline{toc}{chapter}{Appendix}

\begin{Shaded}
\begin{Highlighting}[]
\KeywordTok{library}\NormalTok{(party)}
\KeywordTok{library}\NormalTok{(partykit)}
\KeywordTok{library}\NormalTok{(caret)}
\KeywordTok{library}\NormalTok{(MLmetrics)}
\CommentTok{#install.packages("rattle")}
\KeywordTok{library}\NormalTok{(rattle)}


\CommentTok{# Model Performance }
\OperatorTok{*}\StringTok{  }\NormalTok{Set1 =}\StringTok{ }\NormalTok{Caret}\OperatorTok{:}\StringTok{ }\NormalTok{bagImputed; no additional pre}\OperatorTok{-}\NormalTok{processing  }
\OperatorTok{*}\StringTok{  }\NormalTok{Set2 =}\StringTok{ }\NormalTok{Caret}\OperatorTok{:}\StringTok{ }\NormalTok{bagImputed; PreP }\StringTok{`}\DataTypeTok{method=c('center', 'scale', 'nzv', 'BoxCox')}\StringTok{`}

\NormalTok{#### Train Performance:}
\NormalTok{tbl.perf.train1 }\OperatorTok\StringTok{ }
\StringTok{  }\KeywordTok{kable}\NormalTok{(}\DataTypeTok{caption=}\StringTok{"Train1 Performance"}\NormalTok{, }\DataTypeTok{booktabs=}\NormalTok{T, }\DataTypeTok{digits=}\DecValTok{4}\NormalTok{) }\OperatorTok\StringTok{ }
\StringTok{  }\KeywordTok{kable_styling}\NormalTok{() }
\NormalTok{tbl.perf.train2 }\OperatorTok\StringTok{ }
\StringTok{  }\KeywordTok{kable}\NormalTok{(}\DataTypeTok{caption=}\StringTok{"Train2 Performance"}\NormalTok{, }\DataTypeTok{booktabs=}\NormalTok{T, }\DataTypeTok{digits=}\DecValTok{4}\NormalTok{) }\OperatorTok\StringTok{ }
\StringTok{  }\KeywordTok{kable_styling}\NormalTok{() }

\NormalTok{#### Test Accuracy:}
\NormalTok{tbl.perf.test1 }\OperatorTok\StringTok{ }
\StringTok{  }\KeywordTok{kable}\NormalTok{(}\DataTypeTok{caption=}\StringTok{"Test1 Performance"}\NormalTok{, }\DataTypeTok{booktabs=}\NormalTok{T, }\DataTypeTok{digits=}\DecValTok{4}\NormalTok{) }\OperatorTok\StringTok{ }
\StringTok{  }\KeywordTok{kable_styling}\NormalTok{() }
\NormalTok{tbl.perf.test2 }\OperatorTok\StringTok{ }
\StringTok{  }\KeywordTok{kable}\NormalTok{(}\DataTypeTok{caption=}\StringTok{"Test2 Performance"}\NormalTok{, }\DataTypeTok{booktabs=}\NormalTok{T, }\DataTypeTok{digits=}\DecValTok{4}\NormalTok{) }\OperatorTok\StringTok{ }
\StringTok{  }\KeywordTok{kable_styling}\NormalTok{() }

\CommentTok{# New Model : RPART}

\NormalTok{##Metrics}
\KeywordTok{set.seed}\NormalTok{(}\DecValTok{58677}\NormalTok{)}
\NormalTok{grid_rpart <-}\StringTok{ }\KeywordTok{expand.grid}\NormalTok{(}\DataTypeTok{maxdepth =} \DecValTok{1}\OperatorTok{:}\DecValTok{20}\NormalTok{)}
\NormalTok{fit.rpart <-}\StringTok{ }\KeywordTok{train}\NormalTok{(PH}\OperatorTok{~}\NormalTok{., }\DataTypeTok{data=}\NormalTok{train_trans, }\DataTypeTok{metric=}\StringTok{"RMSE"}\NormalTok{, }\DataTypeTok{method =} \StringTok{"rpart2"}\NormalTok{, }\DataTypeTok{tuneGrid =}\NormalTok{ grid_rpart, }\DataTypeTok{tuneLength=}\NormalTok{tl, }\DataTypeTok{trControl=}\NormalTok{trC)}
\NormalTok{rp.PERF <-}\StringTok{ }\KeywordTok{rbind}\NormalTok{(}\KeywordTok{getTrainPerf}\NormalTok{(fit.rpart))}
\NormalTok{rp.MAPE <-}\StringTok{ }\KeywordTok{cbind}\NormalTok{(}\DataTypeTok{MAPE =} \KeywordTok{MAPE}\NormalTok{(fit.rpart}\OperatorTok{$}\NormalTok{pred}\OperatorTok{$}\NormalTok{pred, fit.rpart}\OperatorTok{$}\NormalTok{pred}\OperatorTok{$}\NormalTok{obs))}
\NormalTok{rp.ACC <-}\StringTok{ }\KeywordTok{cbind}\NormalTok{(rp.MAPE, rp.PERF)}
\NormalTok{fit.rpart2 <-}\StringTok{ }\KeywordTok{train}\NormalTok{(PH}\OperatorTok{~}\NormalTok{., }\DataTypeTok{data=}\NormalTok{train_trans, }\DataTypeTok{metric=}\StringTok{"RMSE"}\NormalTok{, }\DataTypeTok{method =} \StringTok{"rpart"}\NormalTok{, }\DataTypeTok{tuneLength=}\NormalTok{tl, }\DataTypeTok{trControl=}\NormalTok{trC)}
\NormalTok{rp.PERF2 <-}\StringTok{ }\KeywordTok{rbind}\NormalTok{(}\KeywordTok{getTrainPerf}\NormalTok{(fit.rpart2))}
\NormalTok{rp.MAPE2 <-}\StringTok{ }\KeywordTok{cbind}\NormalTok{(}\DataTypeTok{MAPE =} \KeywordTok{MAPE}\NormalTok{(fit.rpart2}\OperatorTok{$}\NormalTok{pred}\OperatorTok{$}\NormalTok{pred, fit.rpart2}\OperatorTok{$}\NormalTok{pred}\OperatorTok{$}\NormalTok{obs))}
\NormalTok{rp.ACC2 <-}\StringTok{ }\KeywordTok{cbind}\NormalTok{(rp.MAPE2, rp.PERF2)}
\KeywordTok{rbind}\NormalTok{(rp.ACC, rp.ACC2) }\OperatorTok\StringTok{ }\KeywordTok{kable}\NormalTok{() }\OperatorTok\StringTok{ }\KeywordTok{kable_styling}\NormalTok{()}

\NormalTok{##Tune Grid}
\NormalTok{p1<-}\KeywordTok{ggplot}\NormalTok{(fit.rpart) }\OperatorTok{+}\StringTok{ }\KeywordTok{theme_bw}\NormalTok{() }\OperatorTok{+}\StringTok{ }\KeywordTok{theme}\NormalTok{() }\OperatorTok{+}\StringTok{ }\KeywordTok{labs}\NormalTok{(}\DataTypeTok{title =} \StringTok{"Cart1 CV TUNE GRID"}\NormalTok{)}
\NormalTok{p2<-}\KeywordTok{ggplot}\NormalTok{(fit.rpart2) }\OperatorTok{+}\StringTok{ }\KeywordTok{theme_bw}\NormalTok{() }\OperatorTok{+}\StringTok{ }\KeywordTok{theme}\NormalTok{() }\OperatorTok{+}\StringTok{ }\KeywordTok{labs}\NormalTok{(}\DataTypeTok{title =} \StringTok{"Cart2 CV TUNE GRID"}\NormalTok{)}
\NormalTok{gridExtra}\OperatorTok{::}\KeywordTok{grid.arrange}\NormalTok{(p1, p2, }\DataTypeTok{nrow=}\DecValTok{1}\NormalTok{)}

\NormalTok{##Cool Plot}
\KeywordTok{fancyRpartPlot}\NormalTok{(fit.rpart}\OperatorTok{$}\NormalTok{finalModel,}
               \DataTypeTok{main =} \OtherTok{NA}\NormalTok{,}
               \DataTypeTok{sub =} \StringTok{"Visualization of Cart1 Trees Influence on pH"}\NormalTok{,}
               \DataTypeTok{caption =} \OtherTok{NA}\NormalTok{,}
               \DataTypeTok{palettes =} \StringTok{"PuBuGn"}\NormalTok{,}
               \DataTypeTok{type =} \DecValTok{2}\NormalTok{) }
\KeywordTok{fancyRpartPlot}\NormalTok{(fit.rpart2}\OperatorTok{$}\NormalTok{finalModel,}
               \DataTypeTok{main =} \OtherTok{NA}\NormalTok{,}
               \DataTypeTok{sub =} \StringTok{"Visualization of Cart2 Trees Influence on pH"}\NormalTok{,}
               \DataTypeTok{caption =} \OtherTok{NA}\NormalTok{,}
               \DataTypeTok{palettes =} \StringTok{"Purples"}\NormalTok{,}
               \DataTypeTok{type =} \DecValTok{2}\NormalTok{) }

\NormalTok{##ALT PLOT}
\NormalTok{rpartY <-}\StringTok{ }\KeywordTok{as.party.rpart}\NormalTok{(fit.rpart}\OperatorTok{$}\NormalTok{finalModel)}
\KeywordTok{plot.party}\NormalTok{(rpartY)}
\NormalTok{rpartY <-}\StringTok{ }\KeywordTok{as.party.rpart}\NormalTok{(fit.rpart2}\OperatorTok{$}\NormalTok{finalModel)}
\KeywordTok{plot.party}\NormalTok{(rpartY)}
\end{Highlighting}
\end{Shaded}


\end{document}
